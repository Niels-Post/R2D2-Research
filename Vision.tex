%! Author = R2D2 Team 3
%! Date = 14/04/2020

% Preamble
\documentclass[11pt]{article}



\title{Een methode voor pijndetectie door middel van Computer Vision, geörienteerd op embedded systemen}

\author{\emph{R2D2 Team 3} \and Otto de Visser \and Niels Post}

% Document
\begin{document}
    \maketitle

    \clearpage
    \renewcommand{\contentsname}{Inhoudsopgave}
    \tableofcontents

    \clearpage


    \section{Samenvatting}


    \section{Inleiding}
    \emph{R2D2 heeft als bedrijf als doel om multifunctionele, modulaire robots te maken voor gebruik in een rampgebied.
    Omdat in een rampgebied vaak slachtoffers zijn die door allerlei oorzaken hevige pijn hebben,
    is het belangrijk dit te indexeren, en hiernaar prioriteiten te stellen.}

    In dit onderzoek willen wij manieren vinden om door middel van Computer Vision in een klein (embedded) systeem pijnniveaus
    te detecteren.
    Vanwege mogelijke beperkte connectiviteit in een rampgebied onderzoeken wij methoden die volledig automatisch,
    zonder netwerkconnectiviteit of zware hardware kunnen werken.

    Tijdens het lezen van dit onderzoek zal de lezer in de eerste plaats meer leren over vision-algoritmes met pijn
    herkennen als doel.
    Hierbij gaan wij specifiek in op de efficiëntie,geschikte hardware, en verwerkingstijden van hardware.
    Hiernaast bespreken wij welke camera’s het meest geschikt zijn voor gebruik bij Computer Vision in het algemeen.
    Uiteindelijk kiezen wij 1 combinatie van een algoritme en camera die het best passen bij de context van een pijn
    detectie module binnen een R2D2 robot.


    Om dit onderzoek te definiëren zullen wij uit gaan van de volgende onderzoekshoofdvraag:\\

    \textbf{Hoofdvraag} \hspace{5} \emph{Welke computer methoden op basis van computer vision zonder gebruik van netwerkconnectiviteit high
    performance hardware bestaan er waarmee je pijn kan kwantificeren in de context van een embedded systeem?}

    \bigskip

    Deze hoofdvraag verdelen wij in de volgende 7 deelvragen:

    \begin{enumerate}
        \item Is het mogelijk om computer vision toe te passen op videobeeld om pijn te kunnen herkennen op het gezicht van een mens?
        \item Hoeveel foto’s per tijdseenheid moeten we minimaal verwerken om zeker te zijn dat we een valide uitspraak kunnen doen over het pijnniveau dat iemand ervaart?
        \item Wat zijn de minimale hardwarevereisten van een systeem dat pijnmetingen door middel van Computer Vision uitvoert?
        \item Wat is de beste camera voor het toepassen van Vision algoritmes die verkrijgbaar is op de markt?
        \item Volgt er een toename in accuraatheid uit het langer analyseren van een frame?
        \item Welke bestaande implementaties van een pijnmeting algoritme zijn er, en wat zijn de gebruiksrechten hiervan?
        \item Wat is de maximaal bereikbare accuraatheid en snelheid bij gebruik van boven onderzochte methoden?
    \end{enumerate}


    \section{Literatuurverkenning}
    In dit hoofdstuk verkennen we de onderzochte literatuur.
    Hierbij beschrijven wij per stuk wat wij verwachten hieruit te halen, of hoe dit ons heeft geholpen met onze onderzoeksvragen.


    \section{Probleemstelling}
    Voor doktoren is het lastig om een objectief pijn vast te stellen.
    Vaak moet een patiënt zelfeen getal tussen de 1 en de 10 geven voor zijn pijn.
    Echter is deze manier van pijn inschatten niet erg objectief, daarnaast is het niet
    altijd mogelijk met de patiënt te communiceren.
    In rampscenario’s zijn er bijvoorbeeld te veel slachtoffers om aan iedereen te vragen hoeveel pijn hij voelt.
    Wel is het belangrijk om te weten hoeveel pijn mensen hebben om ze met meer/minder prioriteit te helpen.
    Hierom, in combinatie met de doelen van het R2D2 bedrijf, bestaat de vraag naar een automatische module die pijn
    detecteert.


    \section{Theorie en hypothese}


    \section{Begrippenlijst}


    \section{Uitvoering}
    Voor deelvraag 1 zullen wij een cross-sectioneel onderzoek uitvoeren.
    Vooraf bepalen wij waaraan zo een algoritme moet voldoen, en stellen hier metrics voor op.
    Vervolgens passenwij deze metrics toe op alle algoritmes, om zo de beste optie te kunnen vinden.
    Voor deelvraag 2 doen wij deskresearch op basis van de bezochte methoden.
    Wij interpreteren uit de methode hoeveel tijd er minimaal nodig is om pijn te detecteren, en vergelijken deze tijden.
    Voor deelvraag 3 doen wij ook deskresearch, maar hier kijken wij naar de gebruiktecomputerkracht van een methode.
    Op basis van een gegeven testopzet van een onderzoekschatten wij de benodigde geheugenruimte en processorkracht in.
    Bij het beantwoorden van deelvraag 4 maken wij deels gebruik van de resultaten vandeelvraag 3.
    De hardware-benodigdheden limiteren namelijk ook de camera.
    Een cameragebruiken met een hogere resolutie dan het systeem aankan heeft namelijk geen zin.
    Verdervergelijken wij bestaande onderzoeken over camerakeuze in Vision systemen.Voor deelvraag 4 doen wij experimenteel onderzoek.
    Wij testen onze implementatie vanpijnherkenning meerdere keren op dezelfde dataset, terwijl wij de analysetijd per framesteeds aanpassen.
    Wij vergelijken hoe goed de accuraatheid van de implementatie was perinstelling van de analysetijd.
    Deelvraag 5 is een deskresearch.
    Wij onderzoeken en beschrijven bestaandeimplementaties van vision-pijnmeting algoritmen.
    Voor deelvraag 6 doen wij experimenteel onderzoek.
    Door meerdere implementaties tevergelijken, kunnen wij de maximale bereikbare waarden ervan vergelijken.


    \section{Resultaten}


    \section{Conclusie en Discussie}


    \section{Evaluatie}


    \section{Aanbevelingen}


    \section{Suggesties voor verder onderzoek}


    \section{Literatuur}


    \section{Bijlagen}


    \section{Referenties}


\end{document}