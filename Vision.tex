%! Author = R2D2 Team 3
%! Date = 14/04/2020

% Preamble
\documentclass[11pt]{article}
\usepackage{hyperref}
\usepackage{natbib}



\title{Een methode voor pijndetectie door middel van Computer Vision, geörienteerd op embedded systemen}

\author{\emph{R2D2 Team 3} \and Otto de Visser \and Niels Post}

% Document
\begin{document}
    \maketitle

    \clearpage
    \renewcommand{\contentsname}{Inhoudsopgave}
    \tableofcontents

    \clearpage


    \section{Samenvatting}


    \section{Inleiding}
    \emph{R2D2 heeft als bedrijf als doel om multifunctionele, modulaire robots te maken voor gebruik in een rampgebied.
    Omdat in een rampgebied vaak slachtoffers zijn die door allerlei oorzaken hevige pijn hebben,
    is het belangrijk dit te indexeren, en hiernaar prioriteiten te stellen.}

    In dit onderzoek willen wij manieren vinden om door middel van Computer Vision in een klein (embedded) systeem pijnniveaus
    te detecteren.
    Vanwege mogelijke beperkte connectiviteit in een rampgebied onderzoeken wij methoden die volledig automatisch,
    zonder netwerkconnectiviteit of zware hardware kunnen werken.

    Tijdens het lezen van dit onderzoek zal de lezer in de eerste plaats meer leren over vision-algoritmes met pijn
    herkennen als doel.
    Hierbij gaan wij specifiek in op de efficiëntie,geschikte hardware, en verwerkingstijden van hardware.
    Hiernaast bespreken wij welke camera’s het meest geschikt zijn voor gebruik bij Computer Vision in het algemeen.
    Uiteindelijk kiezen wij 1 combinatie van een algoritme en camera die het best passen bij de context van een pijn
    detectie module binnen een R2D2 robot.


    Om dit onderzoek te definiëren zullen wij uit gaan van de volgende onderzoekshoofdvraag:\\

    \textbf{Hoofdvraag} \hspace{5pt} \emph{Welke computer methoden op basis van computer vision zonder gebruik van netwerkconnectiviteit high
    performance hardware bestaan er waarmee je pijn kan kwantificeren in de context van een embedded systeem?}

    \bigskip

    Deze hoofdvraag verdelen wij in de volgende 7 deelvragen:

    \begin{enumerate}
        \item Is het mogelijk om computer vision toe te passen op videobeeld om pijn te kunnen herkennen op het gezicht van een mens?
        \item Hoeveel foto’s per tijdseenheid moeten we minimaal verwerken om zeker te zijn dat we een valide uitspraak kunnen doen over het pijnniveau dat iemand ervaart?
        \item Wat zijn de minimale hardwarevereisten van een systeem dat pijnmetingen door middel van Computer Vision uitvoert?
        \item Wat is de beste camera voor het toepassen van Vision algoritmes die verkrijgbaar is op de markt?
        \item Volgt er een toename in accuraatheid uit het langer analyseren van een frame?
        \item Welke bestaande implementaties van een pijnmeting algoritme zijn er, en wat zijn de gebruiksrechten hiervan?
        \item Wat is de maximaal bereikbare accuraatheid en snelheid bij gebruik van boven onderzochte methoden?
    \end{enumerate}


    \section{Literatuurverkenning}
    In dit hoofdstuk verkennen we de onderzochte literatuur.
    Hierbij beschrijven wij per stuk wat wij verwachten hieruit te halen, of hoe dit ons heeft geholpen met onze onderzoeksvragen.


    \section{Probleemstelling}
    Voor doktoren is het lastig om een objectief pijn vast te stellen.
    Vaak moet een patiënt zelfeen getal tussen de 1 en de 10 geven voor zijn pijn.
    Echter is deze manier van pijn inschatten niet erg objectief, daarnaast is het niet
    altijd mogelijk met de patiënt te communiceren.
    In rampscenario’s zijn er bijvoorbeeld te veel slachtoffers om aan iedereen te vragen hoeveel pijn hij voelt.
    Wel is het belangrijk om te weten hoeveel pijn mensen hebben om ze met meer/minder prioriteit te helpen.
    Hierom, in combinatie met de doelen van het R2D2 bedrijf, bestaat de vraag naar een automatische module die pijn
    detecteert.


    \section{Theorie en hypothese}


    \section{Begrippenlijst}


    \section{Uitvoering}
    Voor deelvraag 1 zullen wij een cross-sectioneel onderzoek uitvoeren.
    Vooraf bepalen wij waaraan zo een algoritme moet voldoen, en stellen hier metrics voor op.
    Vervolgens passen wij deze metrics toe op alle algoritmes, om zo de beste optie te kunnen vinden.

    Voor deelvraag 2 doen wij deskresearch op basis van de bezochte methoden.
    Wij interpreteren uit de methode hoeveel tijd er minimaal nodig is om pijn te detecteren, en vergelijken deze tijden.

    Voor deelvraag 3 doen wij ook deskresearch, maar hier kijken wij naar de gebruikte computerkracht van een methode.
    Op basis van een gegeven testopzet van een onderzoek schatten wij de benodigde geheugenruimte en processorkracht in.

    Bij het beantwoorden van deelvraag 4 maken wij deels gebruik van de resultaten van deelvraag 3.
    De hardware-benodigdheden limiteren namelijk ook de camera.
    Een camera gebruiken met een hogere resolutie dan het systeem aan kan heeft namelijk geen zin.
    Verder vergelijken wij bestaande onderzoeken over camerakeuze in Vision systemen.

    Voor deelvraag 4 doen wij experimenteel onderzoek.
    Wij testen onze implementatie van pijnherkenning meerdere keren op dezelfde dataset, terwijl wij de analysetijd per frame steeds aanpassen.
    Wij vergelijken hoe goed de accuraatheid van de implementatie was per instelling van de analysetijd.

    Deelvraag 5 is een deskresearch.
    Wij onderzoeken en beschrijven bestaande implementaties van vision-pijnmeting algoritmen.

    Voor deelvraag 6 doen wij experimenteel onderzoek.
    Door meerdere implementaties te vergelijken, kunnen wij de maximale bereikbare waarden ervan vergelijken.


    \section{Resultaten}

    \subsection{Deelvraag 1}
    \emph{Is het mogelijk om computer vision toe te passen op videobeeld om pijn te kunnen herkennen op het gezicht van een mens?}

    Wij hebben een aantal velden opgesteld die wij per methode hebben onderzocht. De volgende resultaten zijn hieruit gekomen

    \subsubsection{Methode 1}
    \emph{\citet{werner2014automatic} Automatic pain recognition from video and biomedical signals}

    \paragraph{Korte Omschrijving}
    Het paper omschrijft een bredere methode, maar wij leggen de focus op het gedeelte vision.
    Deze methode detecteert pijn door te kijken naar de positie van bepaalde gezichtseigenschappen ten opzichte van elkaar.
    Zo kan pijn zich bijvoorbeeld uiten in een “samengetrokken” gezicht.

    \paragraph{Van welk principe maakt het gebruik?}
    Relationele positie van gezichtseigenschappen

    \paragraph{Geschatte piek-geheugengebruik per meting}
    Wij gaan uit van een ruime minimum voor gezichtsdetectie.
    Gezichtsdetectie en herkenning functioneren beduidend minder onder een resolutie van 50x50 pixels \citet{boom2006effect}.
    Hier gaan wij dan ook vanuit bij het geheugengebruik.
    Een afbeelding van 50x50 gebruikt +- 7.5kB in ruimte.

    Methoden van facedetection zijn ons niet bekend, hierom is dit moeilijk in te schatten.
    Ervanuitgaande dat er convolutie-operaties uitgevoerd worden op de afbeelding zal de afbeelding hoogst waarschijnlijk op momenten meerdere keren in het geheugen moeten bestaand.
    Hiernaast moet de draaiende programmacode, en eventuele image kernels toegevoegd worden.
    In piek zal dit programma minimaal 50 kB geheugen gebruiken, maar naar alle waarschijnlijkheid meer.

    \paragraph{Geschatte tijdsduur voor 1 conclusie}
    Wij denken dat er minimaal 2 metingen gedaan moeten worden.
    Dit omdat het gezicht vergeleken moet worden, en de samentrekking van het gezicht ten opzichte van de eerste meting bepaald moet worden.

    Op een adequate computer hoeft een meting niet veel langer te duren dan 0,1 seconde.
    Twee metingen zouden dan zo’n 0.2 seconden duren.

    \paragraph{Embedded capaciteit}
    Zoals wij al enigszins hadden verwacht is het niet redelijkerwijs mogelijk dit te draaien op een Arduino.
    Dit is voornamelijk omdat het piek-geheugengebruik te groot is hiervoor.
    Wel is dit prima mogelijk om te draaien op een Raspberry Pi of een mini-pc.

    \paragraph{Maximale geschatte effectieve afstand}
    Dit is voornamelijk afhankelijk van de tijd die wij besteden aan het herkennen.
    In combinatie met een camera met een hoge resolutie kunnen afstanden van vele meters gehaald worden (uitgaande van een minimum gezichtsgrootte van 50pxx50px.
    Hiervoor zou een face detection algoritme wel erg vaak/erg precies uitgevoerd worden, wat de verwerkingstijd flink vergroot.
    De maximale afstand is in dit geval dus afhankelijk van de maximale verwerkingstijd

    \paragraph{Kosten vereiste specifieke apparatuur}
    Voor deze methode is alleen een (mini)-pc en een camera vereist. Een (minimale) combinatie van een Raspberry Pi (€40) en een standaard webcam(€30) zou ongeveer €70,- kosten.

    \paragraph{Haalbaarheid}
    Wij denken dat deze methode voor onze scope haalbaar is, met een paar aanpassingen:
    \begin{itemize}
        \item Deze methode projecteert gezichtseigenschappen op een 3d-model van een gezicht. Hier hebben wij noch ervaring, noch tijd voor om uit te werken.
        Om dit te vermijden zouden wij de aanname doen dat iemand recht in de camera kijkt.
        Zo kunnen wij toch met zekerheid afstanden tussen gezichtseigenschappen meten.
        \item Zoals eerder aangegeven richten wij ons enkel op het vision-gedeelte van dit onderzoek, niet op de biomedische sensoren.
    \end{itemize}

    \paragraph{Mogelijke implementatie}
    Wij maken gebruik van DLib, een c++ library waar een methode voor facial feature detection in opgenomen is.
    Uit een eerste foto/meting extraheren wij een aantal afstanden tussen bepaalde gezichtskenmerken, en slaan deze op.
    Hierna doen wij steeds opnieuw metingen en vergelijken dit met de oorspronkelijke afstandsmetingen,
    hierbij kijkend naar de specifieke veranderingen zoals genoteerd in de onderzochte papers.

    \subsubsection{Methode 2}
    \emph{\citet{Performance_Of_Optical_Flow_Techniques} Performance Of Optical Flow Techniques.}\\
    \emph{\citet{An_Efficient_Algorithm_for_Motion_Detection_Based_Facial_Expression_Recognition_using_Optical_Flow} An Efficient Algorithm for Motion Detection Based Facial Expression Recognition using Optical Flow.}\\
    \emph{\citet{Facial_expression_change_detection_algorithm_using_optical_flow_technique} Facial expression change detection algorithm using optical flow technique.}\\

    \paragraph{Korte Omschrijving}
    Methode 2 draait om de zogenoemde techniek optical flow. 
    Dit is een veel gebruikte techniek om verandering tussen frame in beeld te brengen en te kwantificeren.   

    \paragraph{Van welk principe maakt het gebruik?}
    Deze techniek kijkt naar verschillen tussen frames en probeert de richting van verandering vast te stellen. 
    Er wordt over het gezicht gekeken naar verandering, welke uitgedrukt worden in vectoren.
    Deze worden dan gematcht tegen een database met de verschillende emoties.
    Door dat je verandering in mensen hun gezicht registreerd 

    \paragraph{Geschatte piek-geheugengebruik per meting}


    \paragraph{Geschatte tijdsduur voor 1 conclusie}
    De techniek kan met drie frames al een nauwkurigheid van 83\% opleveren volgens \emph{\citet{An_Efficient_Algorithm_for_Motion_Detection_Based_Facial_Expression_Recognition_using_Optical_Flow}}. 
    Deze drie frames zouden echter wel de verschillende staten van de emotie die uitgedrukt worden moeten in beeld moeten brengen.
    De 

    \paragraph{Embedded capaciteit}

    \paragraph{Maximale geschatte effectieve afstand}

    \paragraph{Kosten vereiste specifieke apparatuur}

    \paragraph{Haalbaarheid}
    Wij denken dat deze methode voor onze scope haalbaar is, met een paar aanpassingen:
    \begin{itemize}
        \item 
        \item 
    \end{itemize}

    \paragraph{Mogelijke implementatie}


    \section{Conclusie en Discussie}
    Test \citet{werner2014automatic}


    \section{Evaluatie}


    \section{Aanbevelingen}


    \section{Suggesties voor verder onderzoek}


    \section{Literatuur}


    \section{Bijlagen}

    \bibliographystyle{apalike}
    \bibliography{Bibliography/Main}

\end{document}


